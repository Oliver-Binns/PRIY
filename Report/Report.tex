\documentclass{UoYCSproject}
\usepackage{graphicx}
\usepackage[nottoc,numbib]{tocbibind}
\usepackage{subcaption}
\usepackage[acronym]{glossaries}
\usepackage{listings}

\addbibresource{Report.bib}
\title{Parallel Programming Tools for Exploring Immune System Development}
\author{Oliver Binns}
\MEng
\date{\today}
\supervisor{Dr. Fiona Polack, Dr. Kieran Alden}
\wordcount{-1}

\abstract{More powerful computers are paving the way for realistic simulations of previously underexplored complex biological systems. As advancements in computing tend towards parallelism and distributed systems, increasingly realistic simulations must take full advantage of this change in order to be computed in a reasonable time period. In this project, we will explore some of the current technologies for doing so and propose new tools for the future.}
 
\begin{document}
\maketitle
\listoffigures
\listoftables
\renewcommand*{\lstlistlistingname}{List of Listings}
\lstlistoflistings

\chapter{Introduction}
\section{Project Overview}
TALK ABOUT MODELING AND THE DESIRE FOR 
\\
Faster AND More Accessible Simulation

\section{Motivation}
%What can be gained from better modelling.. particularly biology..?
%Presentation- see justification slides!
The motivation for this project stems mainly from the new knowledge that could be gained from the ability to create efficient parallel biological simulations more easily.

If hypotheses for *drug testing* can be more easily and extensively tested in silico, this could significantly improve the process by which drugs are developed.
New drugs and cures could be developed more quickly, allowing these to be available more quickly.
The financial burden of drug development (approximately \$2.5bn[citation needed]) could be reduced, freeing up heavily contested funds for additional research.
The use of animal testing could be reduced, and potentially eliminated completely in the long term.

%outside of drug testing, new understanding can..?

\section{Project Aims}
In summary, the aims of this project are to
\begin{enumerate}
	\item Provide a thorough review of the use of simulations as part of Computational Biology, the advantages they provide and the problems that must be overcome for their mass adoption.
	\item Develop a parallel implementation of an existing simulation of Peyer's Patch development and explore any speed increases that can be produced using General Purpose GPU programming.
	\item Establish a firm grounding for the future development of new tools to allow fast, parallel simulations of biological systems to be easily created by non-technical users.
	\begin{enumerate}
		\item Explore the findings of the new implementation and discuss how these can be generalised to new simulations.
		\item Discuss techniques for allowing non-technical users to easily create formal models that can be transformed into new simulation implementations.
	\end{enumerate}
\end{enumerate}

\section{Statement of Ethics}
This project was conducted in accordance with the University of York's code of practice on ethics.
This project does not involve human participants, so guidelines on informed consent and confidentiality will be met. No confidential medical data or personal information has been used during the course of the project development. This project has involved no direct animal participation.
%[Where did the data come from?!], how was the original model created, should this be mentioned?

The simulation of the biological model is for the purpose of developing understanding of applying GPGPU methods to an agent based model of a biological system. It will not be used its current form to publish novel biological findings and does not fully simulate a biological process.%WAIT, is this true?
%Software used and licences? FlameGPU is freely available / open source
%Cuda is proprietary but freely available..
%Hardware.. is this relevant?


\section{Report Structure}
This report details the work done throughout the project and 

Chapter \ref{background} gives a general overview of simulations and the benefits and limitations of their use particularly with regard to computational biology.

Chapter \ref{improvements} explores some of the limitations of simulations in additional detail and proposes future solutions for these.

Chapter \ref{methods} details the development of an improved, inherently parallel, implementation of PPSim.


\chapter{Background}
\label{background}
% Aims are to:
%
% Place original work into the context of existing Literature
% Interpret major issues surrounding the topic
% Describe the relationship of each work to the others under consideration 
% Identify new ways to interpret, and shed light on any gaps in previous research
% Resolve conflicts among seemingly contradictory studies
% Determine which literature makes a significant contribution to your understanding of the topic
% Point your way to further research on your topic

\section{Simulation}
%Background, what are simulations used for generally
Simulations are model-based imitations of a system which feature its key characteristics and behaviours. Computer simulations are used in a wide range of disciplines on applications such as video games, medicine, product development and even nuclear weapons.

The models of systems used in simulations may have a varying amounts of abstraction. Simulations used for teaching will likely have models which remove significant amounts of complexity from the system. Simulations used for video games tend to be as realistic as possible as realism has been shown to produce a higher level of immersion\cite{realism_immersion}, a highly desirable attribute of games.%may aim to remove some complexity from the original system (compute times)

%mention speculation that we live in a simulation? - simulation hypothesis, The Matrix, Elon Musk

This chapter will explain the benefits and current limitations of using simulations and how these have affected this project.

\subsection{Benefits}
\subsubsection{Feasibility}
Exploring computer simulations is often far more feasible than exploring a real world environment. Video games are simulations which may allow players to experience scenarios that they may not otherwise get the opportunity to encounter. For example, car racing games are significantly cheaper and safer than real life racing.
%video games, clearly easier/cheaper/safe for someone to play a war/car simulation than experience this in real life

Within scientific discovery, simulations may be used as an alternative to real-world testing or to complement it. Often real-world scientific testing may not be feasible for a number of reasons. Simulating the aerodynamics of new car designs virtually is far quicker and cheaper than creating multiple different prototypes for physical tests. Morality may be a factor, animal testing for cosmetic products or medicine is a good example of this. With nuclear weapons, legality is a key issue, as weapon testing is banned under a number of global treaties\cite{partial_nuclear_test_ban_treaty, threshold_test_ban_treaty, comprehensive_nuclear_test_ban_treaty}. Finally, these real-world tests may be too dangerous to perform, such as in the case of invasive medical examinations. %SO WHAT? %POINT, EVIDENCE, EXPLANATION
%In all of these examples, simulation can be used instead of or in order to reduce the amount of testing required.
%can scale be a factor? simulations can quickly model across numerous independent variables - see environmental control


\subsubsection{Reducing Complexity}
Reducing complexity through abstraction allows better understanding of the system to be gained as the complexity may initially be overwhelming.
%Removing complexity can help with understanding by not 

\subsubsection{Environmental Control}
Computer simulations allow for the environment to be more easily controlled. The ability to adjust external factors and independent variables that may affect the system on demand can be particularly useful. This ability can be used for illustrating why different variables in system processes are important.

Time can be manipulated to show system processes at more reasonable time scales. A chemical process that takes a fraction of a second can be slowed down to ensure that it can be seen.
%speed up, reverse the simulation

%easier to measure?

%can add additional tests in later if desired, and re-run the simulation

%clearly simulations may be more feasible but
%WHY are they a good alternative to real world testing?!
%accurate/realistic results produced
	%when is this the case? always? when sufficient knowledge is known?
	%relate this to laws of Physics, theoretical biology?

\subsubsection{Visualisation}
One of the biggest benefits of simulation is the ability to graphically visualise a system or its constituent parts. Using simulation for visualisation has number of benefits over attempting to demonstrate real world systems. Several of these benefits translate from the previous two sections- they are often more feasible to explore than a real world environment and featuring a reduced complexity often allows the key concepts to be understood without overwhelming the user.

Simulation can be particularly useful for visualising concepts for education. 

Particularly immersive visualisations can also be produced using virtual and augmented reality. The Virtuali-Tee is an educational tool that uses simulation and augmented reality to provide a view at the body's internal organs\cite{curiscope}. Little Journey aims to reduce kids' anxiety about surgery by providing a realistic tour of their hospital ward given by animated cartoon characters\cite{little_journey}.

\subsection{Limitations and Constraints}
%ARE THERE ANY CONS?
%SHOULD simulations be used by everyone?
%Why are they not?
While simulations seem very useful across a wide number of fields, there are some significant limitations as to where and how they can be used.

\subsubsection{Insufficient Domain Knowledge}
\label{domain_knowledge}
A simulation is based on a model of a system. A model is an abstraction from reality representing only the necessary key characteristics and behaviours of the system. A lack of knowledge regarding the domain of the simulation is one of the most significant constraints regarding its implementation. If this is the case, the model produced may be incorrect or abstractions may remove necessary detail required for the system to function as expected.%Incorrect models can also be a result of incorrect knowledge of the domain.

For complex systems, having too many abstractions from the original domain may invalidate the model and produce incorect results[CITATION NEEDED].

\subsubsection{Compute Power}
Complex models with too few abstractions from their domain may require significant computing power to simulate. Additional abstractions may not be possible as they may invalidate the model. In these cases, cutting edge hardware may be needed for the simulation to be run in an acceptable time. 


Powerful hardware is expensive to access, so this may be a significant constraint on the ability to simulate.

%Agent-Based Modelling, which will be introduced in detail in Section \ref{abm}, models the system as a number of autonomous individuals. This technique requires significantly more compute power than the traditional top-down simulations.

%Talk about the need to speed up
%evidence behind this--!
	%machine learning paper- drive to better capture biological complexity
	%video games, better graphics and more detail
	%other examples?

%only get data on variables that have been modelled, whole system is likely not modelled, see TWA Flight 800...:
%The NTSB deemed a physical simulation appropriate as they were not convinced an available computer model would confirm the true cause of the accident.


\subsubsection{Skills Shortage}
\label{skills_shortage}
The previous section discussed reliance that complex simulations have on advanced hardware. However, advanced hardware alone will not necessarily allow a simulation to compute in a reasonable amount of time. 
%mention the concurrency- free lunch is over article findings
Often, and particularly with the increasingly parallel modern architectures, the simulation code must be tailored to take advantage of the computing power available. Efficient parallel programs that do this rely on the availability of experienced programmers. May recent studies have highlighted the existance of significant computer science skills shortages, across the world\cite{digital_skills_uk, microsoft_blog}. These skills shortages may be significantly limiting the possibility for cross-disciplinary work to utilise fast, advanced simulations.

%Computer science degrees, and related courses and apprenticeships, are proving less popular in recent years
%These are worrying trends considering the demand for digital skills by employers
	% - DIGITAL SKILLS for the UK ECONOMY, A report by ECORYS UK, JANUARY 2016

%Put simply, our nation faces an increasing shortage of individuals with the skills necessary to create and deploy the next generation of information technology.
%Despite the growing importance of computer science, it is only taught in a small percentage of U.S. schools. 
	%Microsoft Corporate Blogs, https://blogs.microsoft.com/on-the-issues/2012/12/12/investing-in-american-innovation-and-the-next-generation/



%As computers are getting easier to use, users are beginning to expect this even from advanced tools
%some tools exist for non-technical users-
	%FlameGPU - simplifies parallisation but still requires indepth programming knowledge
	%Other ongoing project under Kieran's supervision (GUI tool?)

\subsubsection{Bugs}
As with any form of computer program, mistakes can be made causing bugs to be present in the simulation code. Bugs may cause the simulation to be incorrect meaning any hypotheses and results are based on incorrect data.

This is linked to, but not the same as, having insufficent domain knowledge. Both of these limitations will cause the simulations fail silently, produce incorrect results with no immediately obvious failure[CITE]. However, while these problems are specific to simulation, they are not dissimilar from the issues that can occur from poorly designed real-world testing. %as such, these limitations are no reason to dismiss simulation by themselves

If the simulation needs to be safety-critical, developing it using formal methods and refinement may be a good way to ensure that no bugs are present in the code.

\subsection{Agent-Based Modelling}
\label{abm}
%discuss types of simulation..
%What is ABM?
%WHY IS AGENT BASED MODELLING THE MOST RELEVANT FOR THIS PROJECT

%particularly well suited for these cell-based simulations as can be visualised..

\section{Biological Simulations}
Computational Biology is a relatively new field of study that has been growing signficantly over the last decade[CITATION NEEDED].

Within Biology, simulation is often required as an alternative to invasive medical testing/animal testing


Simulations have even been proposed as a method for exploring a potential set of first principles and mathematics that are specific to biology which could even constitute a new subject- theoretical biology\cite{rise_article}.

\subsection{Existing Simulations}
%Previous implementations of simulations and their problems:
%In order to achieve mass-market userbase, they must be:
%EASIER TO MAKE - less technical users can work on them
%FASTER TO RUN. - more feasible, to use in experiments, more complex sims can be made

\subsubsection{Cell Dynamics}


\subsection{PPSim}
\label{ppsim}
This project focuses on simulation as a tool for exploring biological systems at cell level. It uses the existing simulation of Peyer's Patch\cite{kieran_thesis} and attempts to use parallel computer architectures in order to speed this simulation up. %proof of concept- what new knowledge has PPSim created so far?

Finally, I will propose a new tool, which builds on existing work in order to make this power available to non-technical users? or will I, probably not enough time for this!

\chapter{Improving Simulations}
\label{improvements}
%Section on what we need to support complex simulations
%link this to the previous chapter- discussion on current constraints on simulation:
Significant constraints on when simulations can be used were discussed in the previous chapter. In order for simulations to become more widely adopted, some of these constraints must be overcome.
%In particular, this project will focus on overcoming the problems provided by the current computing skills shortage and the requirement to take maximum advantage of the available computing power.
%The skills shortage can be overcome by providing tools that allow non-technical users to create advanced simulations.

As these simulations becoming more and more prevalent: tools need developing to allow less technical users to create these simulations easier and run faster...

\section{Ease of Creation}
Ease of creation (and maintenance) is an important feature for future simulation particularly due to the aforementioned computer science skills shortage (Section \ref{skills_shortage}).

%difficult to create a framework that allows for easy modelling of BOTH state and interaction between agents[CITE NEEDED?, no but seriously]

%person-centric healthcare, custom simulations may be needed

%biocellion

\subsection{Flexible Modelling}
Flexible Modelling tools could be a good method for allowing new simulations to be created more easily. Using flexible modelling, non-technical users would be able to create sketches which can be automatically processed by tools into formal models and prototype metamodels\cite{Paige2017}. FlexiSketch is a good example of this and provides a good tool for creating models and metamodels for software development\cite{flexisketch}.

\section{Speed Up}
%discussed compute power above, relate this specifically to biology now!
%this speed up is needed to keep pace with the additional complexity required.
%more complexity = more realistic = more data to process and more statistical analyses to evaluate = longer to run
%additional complexity:
	%person centric healthcare- simulations must be customisable [machine learning paper]

\subsection{Machine Learning}
%Talk about Kieran's original thesis and how long it took to run, and to develop the results, etc.
A solution to the speed problem that has been proposed recently is to use machine learning on a small set of results to produce 
%Kieran Machine Learning
This has problems in that...
likely affected by Standard Machine Learning issues? Bias? Overfitting?

Already an ongoing area of research.. 


\subsection{Parallelism}
%Trade-off, Peak Performance vs Programming Ease, this can be related to the previous section
%Complex to program, need to bear in mind:
	%Radius of the agent messages, data transfer cost is HIGH
	%Mindful of multi-thread data manipulation

Parallelism fundamentally changes the game and allows computers to keep following Moore's law even has engineers are struggling to make transisters ever smaller and smaller\cite{concurrency_revolution}. As modern computers tend further towards parallelism to keep providing the speed-ups that have been inherent in the industry over recent years, new parallel algorithms need developing in order to take full advantage of the computing power available.

%CPUs vs GPUs
\subsubsection{CPU vs GPU}
Modern computers provide two main methods for parallelising code.
We are building on a previous project\cite{phil_diss} which layed the groudwork for this.
This previous project outlined the choice between CPU and GPU parallelism and makes the case for exploring GPUs- simplisitically put, this is due to the significantly greater speed ups that can be achieved.

%MENTION DIFFICULTY GETTING ACCESS TO A GPU FOR TESTING!

%partitioning is particularly effective if the vast majority of space is unpopulated


\subsubsection{OpenGL vs CUDA}
A previous project compared OpenGL and CUDA


\subsubsection{Agent Based Modelling}
ABM is particularly well suited to parallelism as each agent makes its own decisions.
Communication between individual agents can be difficult to implement in parallel but parallel communication is by no means limited to ABM.


Mention Flame (traditional) is an attempt to make simulations more accessible via ABM.
%Flexible Large Scale Agent Modelling Environment
%http://flame.ac.uk

\subsection{FLAME GPU}
FLAME GPU is an extension of the original version of FLAME, where the simulations that are created are compiled down to CUDA code. This means the simulation can take full advantage of the significant power of modern NVIDIA GPUs.
%overview of FlameGPU
%http://www.flamegpu.com

%x-machine, etc.
%flame is built on top of CUDA, why?

%using a framework like FLAME
	%relies on FLAME being supported into the future
		%the framework is open source, so hopefully this shouldn't be an issue
	%simulations can always be certain of taking advantage of the latest hardware and portability features that FLAME provides
		%As flame is updated, existing simulations may get faster, taking advantage of newer hardware
		%simulations do not need to be reimplemented to do this


\chapter{Methods}
\label{methods}
\section{The Domain Model}
The domain model is taken from the existing sequential simulation of Peyer's Patch development described in Section \ref{ppsim}.

\section{The Platform Model}
OPTIONS:

Custom Code?\cite{phil_diss}
	not well tested
	less easy to update to support new tools and hardware, new CUDA GPUs
		Each custom code simulation must be updated separately


model based -> FlameGPU\cite{flame_keratinocyte}
	restricted to the framework limitations
		can cells migrate into the system?!

\subsection{Testing}
Talk about how the model was tested to ensure correctness\\
Mention missing link in (incorrect) model from Kieran's paper 

\section{Tools}
%what tools were used?
release 1.5 of FLAME GPU- not yet released
mention VCS- git not good for XML model files as it doesn't understand the model syntax

\chapter{Results and Evaluation}
\section{Findings}
\cite{statistical_tests} could be important for evaluating the performance of FlameGPU against original PPSim


\section{Conclusion}

\section{Further Work}
\subsection{Hardware Availability}
%OpenGL could be something to discuss here
%dismissed by Phil, but further exploration could be good!
%ANDY is working on this
%Talk about difficulty getting access to the necessary NVIDIA hardware

\subsection{Software Generalisibility}%IS THIS A WORD?
%Expand project for more powerful cell based simulation
%allow access to non-technical users
%flexible modelling


\printbibliography

\chapter{Appendix}
\section{Simulation Parameters}


\section{Cell Data Structures}
 
\end{document}