\documentclass{UoYCSproject}
\usepackage{graphicx}
\usepackage[nottoc,numbib]{tocbibind}
\usepackage{subcaption}

\addbibresource{Report.bib}
\title{Parallel Programming Tools for Exploring Immune System Development}
\author{Oliver Binns}
\MEng
\date{\today}
\supervisor{Dr. Kieran Alden, Dr. Fiona Polack}
\wordcount{-1}

\abstract{More powerful computers are paving the way for realistic simulations of previously underexplored biological systems. As advancements in computing tend increasingly towards parallelism and distributed systems, these ever more complex simulations must take full advantage of this change in order to be computed in a reasonable time. In this project, we will explore some of the current technologies for doing so and develop new tools for the future.}
 
\begin{document}
\maketitle
\listoffigures
\listoftables
%\renewcommand*{\lstlistlistingname}{List of Listings}
%\lstlistoflistings

\chapter{Introduction}
\section{Project Overview} %Taken from PHIL- Is this correct for my project?!
This project has been carried out with regard to the set of ethical guidelines set out by the University of York. This project does not involve human participants, so guidelines on informed consent and confidentiality will be met. The simulation of the biological model is for the purpose of developing understanding of applying GPGPU methods to an agent based model. It will not be used to publish results, in its current form and does not simulate a biological process.


\section{Ethics}
No confidential medical data or personal information has been used during the course of the project development.
This project has involved no animal or human participation.

\section{Report Structure}
Lorem ipsum dolor.

\chapter{Literature Review}
% Aims are to:
%
% Place original work into the context of existing Literature
% Interpret major issues surrounding the topic
% Describe the relationship of each work to the others under consideration 
% Identify new ways to interpret, and shed light on any gaps in previous research
% Resolve conflicts among seemingly contradictory studies
% Determine which literature makes a significant contribution to your understanding of the topic
% Point your way to further research on your topic

%SPLIT INTO SECTIONS.. Agent Based Modelling, GPUs for Agent Based Modelling, Understanding GPGPUs and CUDA, etc.

\section{Computer Simulation}
\section{Background}
%Background, what are simulations used for generally
Simulations are used in a wide range of disciplines on applications such as .. nuclear weapons, medical.., video games, product development.

Simulations may be used alongside existing real-world testing or an alternative to it.
The are generally used when extensive real-world testing may not be feasible for a number of reasons including time, cost, danger (nuclear weapons) or scale.

Simulations have even been proposed as a method for exploring a potential set of first principles and mathematics that are specific to biology which could even constitute a new subject- theoretical biology.\cite{rise_article}

%ARE THERE ANY CONS?
%SHOULD simulations be used by everyone?

%Why are they not?
%Computer Science skills shortage
%Computers are getting easier to use, users are beginning to expect this even from advanced tools

\section{Existing Simulations}
%Previous implementations of simulations and their problems:
%In order to achieve mass-market userbase, they must be:
%EASIER TO MAKE - less technical users can work on them
%FASTER TO RUN. - more feasible, to use in experiments, more complex sims can be made


%Talk about Kieran's original thesis and how long it took to run, and to develop the results, etc.
A solution to the speed problem that has been proposed recently is to use machine learning on a small set of results to produce 
%Kieran Machine Learning
This has problems in that...
Standard Machine Learning issues? Bias? Overfitting?

%Section on what we need to support complex simulations
As these simulations becoming more and more prevalent: tools need developing to allow less technical users to create these simulations easier and run faster...


This project focuses on simulation as a tool for exploring biological systems at cell level. It uses the existing simulation of Peyer's Patch\cite{kieran_thesis} and attempts to use parallel computer architectures in order to speed this simulation up. Finally, I will propose a new tool, which builds on existing work in order to make this power available to non-technical users?

\section{Parallelism}
Parallelism fundamentally changes the game and allows computers to keep following Moore's law even has engineers are struggling to make transisters ever smaller and smaller.[citation needed]

As modern computers tend further towards parallelism to keep providing the speed-ups that have been inherent in the industry over recent years, new parallel algorithms need developing in order to take full advantage of the computing power available. 



\section{Understanding GPGPUs and CUDA}

\section{Agent Based Modelling}
Mention Flame (traditional) is a start..

\section{ABM for GPUs}
%WHY USE GPUs?

%Pros/Cons
%Talk about the difficulty getting access to particular hardware needed

\chapter{The Experiment}
\section{The Domain Model}
The domain model is taken from an existing simulation...

\section{The Platform Model}
OPTIONS:
Custom Code?\cite{phil_diss}
model based -> FlameGPU\cite{flame_keratinocyte}

\chapter{Results and Evaluation}
\section{Findings}
\section{Conclusion}

\section{Further Work}
\subsection{Hardware Availability}
%OpenGL could be something to discuss here
%dismissed by Phil, but further exploration could be good!
%Talk about difficulty getting access to the necessary NVIDIA hardware

\subsection{Software Generalisibility}%IS THIS A WORD?

%Expand project for more powerful cell based simulation
%allow access to non-technical users


\printbibliography

\chapter{Appendix}
\section{Simulation Parameters}
\section{Cell Data Structures}
 
\end{document}